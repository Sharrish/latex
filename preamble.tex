% настройка полей документа
\usepackage{geometry}
\geometry{left=3cm}
\geometry{right=1.5cm}
\geometry{top=2cm}
\geometry{bottom=2cm}


% настройка кодировки, шрифтов и русского языка
\usepackage{fontspec}
\usepackage{polyglossia}


% настройки polyglossia
\setdefaultlanguage{russian}
\setotherlanguage{english}


% основной шрифт документа (Антон сказал, что для диплома нужен такой)
\setmainfont{CMU Serif} 


% для заголовков
\usepackage{caption} 
\usepackage{titlesec}
\usepackage[dotinlabels]{titletoc}


% рабочие ссылки в документе
\usepackage{hyperref}
\usepackage{xcolor} % для цветного текста

% настройка различных интервалов
\linespread{1.3} % полуторный межстрочный интервал
\parindent 1.3cm % абзацный отступ (размер красной строки)
\setlength{\parskip}{0.115cm} % расстояние между абзацами (влияет на содержание)


% Для интервала названия дипломной работы и других пользовательских интервалов
\usepackage{setspace}   


% внесение titlepage в учёт счётчика страниц
\makeatletter
\renewenvironment{titlepage} {
 \thispagestyle{empty}
}
\makeatother


% первый абзац секции с красной строки
\PolyglossiaSetup{russian}{indentfirst=true}


% локализация "Содержание" - по центру
\addto\captionsrussian{
  \renewcommand{\contentsname}{\centerline{Содержание}}
}


% subsubsubsection типа "1.2.3.4 Анализ", который отображается в тексте, но не в содержании
\setcounter{secnumdepth}{4}
\titleformat{\paragraph}
{\normalfont\normalsize\bfseries}{\theparagraph}{1em}{}
\titlespacing*{\paragraph}
{0pt}{3.25ex plus 1ex minus .2ex}{1.5ex plus .2ex}


% Пути к каталогам с изображениями
\usepackage{graphicx} % Вставка картинок и дополнений
\graphicspath{{diploma/images}} % % путь к каталогу с рисунками


% оформления подписи рисунка
\captionsetup[figure]{labelsep = period}


% список литературы
\bibliographystyle{diploma/templates/gost-numeric.bbx}
\usepackage{csquotes}
\usepackage[]{biblatex}

% перечень использованных источников
\addbibresource{diploma/refs.bib}

% определение цвета, который будет использоваться в цитировании литературы
\definecolor{my_dark_green}{RGB}{34, 139, 34}

% настройка ссылок и метаданных документа
\hypersetup{unicode=true,colorlinks=true,linkcolor=black,citecolor=my_dark_green,filecolor=magenta,urlcolor=cyan,        		       
    pdftitle={\docname},   	    
    pdfauthor={\studentname},      
    pdfsubject={\subject},      		        
    pdfcreator={\studentname}, 	       
    pdfproducer={Overleaf}, 		     
    pdfkeywords={\subject}
}

